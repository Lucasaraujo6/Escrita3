O presente trabalho propõe um novo modelo matemático para resolução do problema de roteamento e agendamento de veículos para a entrega de concreto pronto com caminhões betoneira, considerando um contexto de múltiplas plantas, frota de veículos homogêneos, plantas sem restrição de capacidade e janelas de tempo para atendimento dos clientes. O modelo proposto se destaca pela resolução de problemas grandes com tempo de execução consideravelmente inferior. Para tanto, o modelo proposto é baseado em um modelo de programação linear inteira mista. O modelo é validado por meio de instâncias utilizadas na literatura, com comparação dos resultados, demonstrando sua eficácia e eficiência na resolução do problema. Os resultados obtidos indicam que o modelo proposto é capaz de resolver problemas de grande porte em tempo hábil, fornecendo soluções de alta qualidade e comprovando sua superioridade em relação aos métodos disponíveis na literatura. O modelo proposto contribui para a otimização dos processos de entrega de concreto pronto, reduzindo custos operacionais e aumentando a eficiência logística das empresas cimenteiras.