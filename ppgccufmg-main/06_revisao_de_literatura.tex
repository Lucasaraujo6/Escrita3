\chapter{Revisão de Literatura}
O problema da entrega de concreto ({\textit concrete delivery problem} - CDP) pode ser considerado uma variação do roteamento de veículos ou {\textit vehicle routing problem} (VRP). Suas principais diferenças são o fato do CDP ser multi-plantas e multi-viagem, onde cada veículo
realiza apenas uma entrega  por viagem, retornando à planta de origem,
o que implica que um mesmo cliente possa ser visitado várias vezes
por veículos diferentes até o atendimento total da demanda.
Em função da natureza do concreto pronto, janelas de tempo de entrega
devem ser respeitadas para que não se inviabilizem uma obra da ICC.

\textcolor{blue}{citar duas revisões recentes de VRP e MULTI TRIP VRP}

Na revisão de literatura, é dada ênfase nos métodos de solução utilizados pelos autores e nas características das instâncias, a fim de ressaltar a novidade da solução apresentada no presente trabalho e a diferença de complexidade das instâncias comuns na literatura para com a aqui adotada.

Foram analisados vários trabalhos que atuam diretamente na solução do CDP, dos quais grande maioria fez uso de dados provindos de aplicações reais, sem fornecer as instâncias para reprodução, e apenas \cite{tabref1}  e \cite{kinable} deixam claro quais são os dados ou como os dados foram gerados. Além dos trabalhos que buscam resolver o problema, \cite{dados} disponibilizam conjunto de dados com 263 instâncias do CDP para acesso público, porém armazenam os dados com extensões lidas por softawres comerciais, o que dificulta o acesso. 

Todos os artigos utilizados como referência fazem uso de um modelo, seja para validar os resultados das heurísticas, seja para resolver o problema no qual o modelo foi aplicado. Nota-se que, nas instâncias dos trabalhos nos quais não há a presença de heurísticas, há apenas uma planta, o que possibilita uma redução considerável da complexidade do problema e facilita sua resolução com resolvedores. 

\cite{tabref1} apresentam um modelo de programação inteira que maximiza o valor ponderado dos pedidos atendidos e consideram um caso especial do problema que pode ser resolvido em tempo polinomial por um algoritmo de fluxo de custo mínimo. Eles adotam em seu artigo a frota homogênea com quantidade de veículos variando enre 2 e 4, quantidade de clientes variando entre 20 e 70 com demanda sendo de apenas uma viagem para cada e com testes para 2 e 3 plantas.  

\cite{tabref2} desenvolvem um modelo baseado em fluxo de rede. Suas instâncias consideram frotas heterogêneas dão abertura para que durante a execução sejam determinadas a demanda, o tamanho da frota, a capacidade e outras informações. Dessa forma, o leitor não tem acesso a essas informações sobre o problema.

\cite{tabref3} desenvolvem um modelo do problema e propõem uma abordagem meta-heu\-rís\-ti\-ca baseada em um algoritmo genético híbrido combinado com heurísticas construtivas. Tomando uma instância real como referência geram instâncias aleatórias com o mesmo tamanho de frota, com 49 veículos, clientes, com 71, e viagens distribuídas entre esses clientes, com 258, considerando apenas uma planta e frota homogênea.

\cite{tabref4} desenvolvem um modelo e, para casos reais, um método de solução. Trabalham com instâncias com 4 ou 5 clientes, apenas uma planta, frota homogênea, viagens na faixa de 97 a 114 viagens por dia e um valor fixo de 45 veículos. 

\cite{asbach} formulam um modelo geral do problema e geram um modelo de fluxo de rede que origina o modelo de programaçao inteira mista. Posteriormente adotam uma busca local. Suas instâncias contam com seis grupos de frotas, nas quais as plantas variam entre 7 e 16 unidades, a quantidade de veículos  disponíveis assume valores enter 74 e 162, são considerados de 22 a 511 clientes com demanda média de 2449,82 m³ de concreto por cliente, resultando em instâncias com a quantidade de viagens por dia entre de 87 a 1071. 

\cite{tabref6} criam um modelo e o utilizam junto à heurística VNS para resolver o problema de forma híbrida. Segundo os autores, as abordagens seguem um esquema de local branch. Os autores fazem o uso de dados de uma companhia real, companhia essa que conta com 31 veículos capazes de transportar concreto, uma quantidade de clientes que varia 13 a 76 por dia com uma média de 42,9 clientes por dia e uma demanda de concreto por cliente variando de 1 a 133 m³, com média de 514,39 m³ por dia. 

\cite{tabref7} formulam as operações de despacho de caminhões de concreto (RMC) como um problema de job shop com recirculação, que inclui janelas de tempo e postergação de demanda para o dia seguinte em um modelo de programação multiobjetivo. Suas instâncias consideram 27 caminhões, de 10 a 12 clientes, apenas uma planta e um total de 66 a 156 viagens por dia.

\cite{tabref8} desenvolvem um modelo de fluxo de rede e um método de solução que incorpora uma técnica de decomposição com o solver. O trabalho possui informações referentes à quantidade de veículos, de 80 a 160 unidades, e de plantas, com 4 unidades, porém deixa a faixa de demanda e a quantidade de clientes ser determinado pelo usuário e não apresenta os valores no trabalho.

\cite{kinable} reconhecem que a falta de dados de referência disponíveis publicamente inibem uma comparação mútua das abordagens do problema em enfoque. Suas abordagens incluem algoritmos exatos e heurísticos. Apresentam um Modelo de Programação Inteira Mista (MIP) e um modelo de Programação de Restrições. Fazem uso das heurísticas "Steepest Descent and best fit" e "Fix-and-optimize heuristic". O trabalho considera de 2 a 20 caminhões, de 10 a 50 clientes, de 1 a 4 plantas e de 2 a 15 viagens por cliente. Os dados das insTãncias são disponibilizados para reprodução do trabalho.

\cite{tabref10} fazem o uso de um método baseado em fluxo de rede e utilizam um algoritmo genético para solução heurística. Em um trabalho posterior, \cite{tabref12} fazem uso de um modelo de rede de tempo-espaço, que combina produção de RMC e despacho de veículos e criam um algoritmo personalizado para resolução do problema. Ambos os trabalhos consideram uma instância com mesmas características, composta por 8 veículos, 7 clientes e uma única planta, com demanda total de 250m³. São os únicos trabalhos estudados que consideram o uso de caminhões bomba na solução desenvolvida. Os caminhões bomba são utilizados quando o caminhão betoneira não possui a bomba, responsável por fazer a planta do concreto no destino, integrada. O fato de contarem com apenas uma planta contribui para a concentração na programação da entrega de bombas e caminhões.


\cite{tabref11} utilizam o modelo de outro autor e fazem uso de heurísticas de Algoritmo Genético Robusto e Algoritmo Genético Sequencial. Utilizam apenas quatro instâncias com dados reais. Não esclarecem a quantidade de caminhões e clientes, mas estabelecem 4 plantas com uma demanda total de 63 a 197 viagens por dia.


Um dos trabalhos analisados de publicação mais recente é o de \cite{cantu}. Seu trabalho foi escolhido para ser a principal referência pela facilidade didática do modelo, pela quantidade de citações obtidas desde a sua publicação e por terem sido encontrados potenciais avanços frente aos métodos e resultados. O trabalho de \cite{cantu}, conta com instâncias com uma faixa de 25 a 150 clientes, 45 a 300 veículos e de 3 a 5 plantas. Assim como muitos dos que pautam esse artigo, não possui uma descrição completadas instâncias e não descreve informações como como faixa ou média de capacidade dos veículos, capacidade das plantas e, principalmente, quantidade de demanda média por cliente.

Os trabalhos supracitados são os que foram identificados como de maiores alinhamentos metodológicos com o presente, por adotarem modelos matemáticos e métodos de aprimoramento da solução seja com heurísticas ou integração destes com modelos matemáticos. Para um melhor entendimento dos problemas correlatos ao CDP, sugerimos
o leitor interessado o trabalho de \cite{tzanetos} que fazem uma revisão sistemática com uma maior gama de autores e metodologias, mapeando também soluções que se baseiam em simulações e aprendizado de máquina. 
