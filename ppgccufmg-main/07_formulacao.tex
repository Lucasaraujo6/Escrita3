\chapter{Notações, definição e formulação}
\label{sec:model}



\section{Introdu\c{c}\~{a}o ao Problema}

O problema de roteamento de caminh\~{o}es betoneira consiste em atender um conjunto de clientes que precisam receber uma quantidade espec\'{i}fica de concreto em suas instala\c{c}\~{o}es. O transporte \'{e} realizado por caminh\~{o}es com capacidade definida, respeitando janelas de tempo para in\'{i}cio e t\'{e}rmino do descarregamento. Todas as entregas devem ocorrer dentro de um horizonte de planejamento.

O concreto \'{e} fornecido por um conjunto de plantas, que podem ter capacidades de produ\c{c}\~{a}o limitadas ou n\~{a}o. A frota pode ser homog\^{e}nea ou heterog\^{e}nea, e a aloca\c{c}\~{a}o inicial dos caminh\~{o}es varia conforme a abordagem adotada: eles podem iniciar e encerrar o dia em um estacionamento central, partir e retornar ao estacionamento ao fim do dia, ou podem ser distribu\'{i}dos dinamicamente, de forma que o modelo determina onde o ve\'{i}culo iniciar\'{a}, ignorando o tempo de aloca\c{c}\~{a}o dos ve\'{i}culos \`{a}s plantas.

Durante a execu\c{c}\~{a}o, os ve\'{i}culos podem operar sob diferentes pol\'{i}ticas de redistribui\c{c}\~{a}o: eles podem permanecer atrelados a uma \'{u}nica planta durante todo o per\'{i}odo ou serem redistribu\'{i}dos entre plantas conforme necess\'{a}rio. Cada caminh\~{a}o atende apenas um cliente por viagem e pode realizar m\'{u}ltiplas viagens ao longo do dia, respeitando sua capacidade de carga e os tempos de deslocamento entre plantas e clientes. Um cliente pode receber m\'{u}ltiplas viagens.

\section*{Caracteriza\c{c}\~{a}o do Problema}

Dada uma inst\^{a}ncia contendo $I$ plantas, $J$ clientes com janelas de tempo $[a_j, b_j]$, tempos de deslocamento $t_{ij}$, demandas $d_j$, e uma frota de $K$ ve\'{i}culos, existe uma aloca\c{c}\~{a}o de caminh\~{o}es que satisfa\c{c}a todas as restri\c{c}\~{o}es operacionais e atenda a todas as demandas?

Para melhor compreender as complexidades envolvidas nesse problema, analisamos sua rela\c{c}\~{a}o com o problema do caixeiro viajante com desigualdade triangular respeitada ($\Delta$-TSP). Em \cite{asbach}, encontra-se a demonstra\c{c}\~{a}o de que o problema de roteamento de caminh\~{o}es betoneira \'{e} NP-completo. A seguir, apresentamos a adapta\c{c}\~{a}o dessa prova.

\section*{Prova de NP-completude}

Para demonstrar que o problema de entrega de concreto \'{e} NP-completo, provamos que ele pertence \`{a} classe NP e mostramos que existe uma redu\c{c}\~{a}o polinomial da vers\~{a}o de decis\~{a}o do $\Delta$-TSP para o problema de roteamento de caminh\~{o}es betoneira.

\subsection*{Passo 1: O problema pertence a NP}

\'{E} evidente que o problema de roteamento de caminh\~{o}es betoneira pertence \`{a} classe NP, pois, dada uma solu\c{c}\~{a}o candidata, podemos verificar em tempo polinomial se todas as restri\c{c}\~{o}es s\~{a}o satisfeitas.

\subsection*{Passo 2: Redu\c{c}\~{a}o a partir do $\Delta$-TSP}

Considere uma inst\^{a}ncia $P$ do $\Delta$-TSP com $n$ cidades e uma fun\c{c}\~{a}o de custo de viagem $z_{TSP}: \{1, \dots, n\} \times \{1, \dots, n\} \to \mathbb{Z}$. Na vers\~{a}o de decis\~{a}o do $\Delta$-TSP, seja $B$ uma constante e a pergunta do problema de decis\~{a}o \'{e}: "Existe um percurso que passe exatamente uma vez por cada cidade, com custo total menor ou igual a $B$?"

Agora, transformamos esta inst\^{a}ncia $(n, z_{TSP}, B)$ em tempo polinomial em uma inst\^{a}ncia $P'$ do problema de entrega de concreto, conforme segue:

\begin{itemize}
    \item Definimos um conjunto de clientes $C := \{C_1, \dots, C_n\}$, onde cada cliente tem demanda 1, correspondendo \`{a}s cidades do TSP.
    \item O conjunto de centrais de concreto \'{e} definido como $D := \{D_1, \dots, D_n\}$.
    \item Cada central pode atender a qualquer cliente, ou seja, $D(c) := D$ para todo $c \in C$.
    \item H\'{a} apenas um ve\'{i}culo $K_1$ com capacidade $q(K_1) := 1$, ou seja, $K := \{K_1\}$.
    \item Os custos de viagem s\~{a}o definidos conforme $z_{TSP}$, garantindo a equival\^{e}ncia entre os problemas.
\end{itemize}

Se existir um percurso na inst\^{a}ncia $P$ do $\Delta$-TSP com custo menor ou igual a $B$, ent\~{a}o existe uma solu\c{c}\~{a}o para a inst\^{a}ncia transformada $P'$ do problema de entrega de concreto com custo menor ou igual a $B$, e vice-versa. Como a redu\c{c}\~{a}o pode ser feita em tempo polinomial, isso prova que o problema de entrega de concreto \'{e} fortemente NP-completo. \hfill $\square$

\section*{Modelo Comparativo}

A seguir, apresentamos um modelo matem\'{a}tico de benchmark extra\'{i}do da literatura para compara\c{c}\~{a}o com o modelo proposto neste trabalho. Este modelo cl\'{a}ssico considera [descrever as principais caracter\'{i}sticas] e ser\'{a} utilizado para avaliar os ganhos proporcionados pela abordagem proposta.

% Inserir modelo de benchmark aqui

\section*{Formula\c{c}\~{a}o do Modelo Proposto}

Com base na estrutura do problema e nas restri\c{c}\~{o}es descritas, propomos o seguinte modelo matem\'{a}tico:

\begin{table}[ht!]
  \centering
  \begin{tabular}{cl}
  \toprule
  \textbf{Parâmetro} & \textbf{Descrição} \\
  \hline
  I & Conjunto de plantas, $I=\{1,\ldots,n_i\}$\\
  J & Conjunto de clientes, $J = \{1,\ldots,n_j\}$ \\
  K & Conjunto de veículos disponíveis, $K=\{1,\ldots,n_k\}$\\
  $L_j$ & Conjunto de viagens possíveis para o cliente j, $L_j=\{1,\ldots,n_{Lj}\}$\\
  $c_{ij}$ & custo do atendimento do cliente $j\in J$ pela planta $i\in I$ \\
  % q & capacidade dos caminhões \\
  $q$ & capacidade e tempo que os veículos levam para descarregar \\
  $d_{j}$ & demanda total do cliente j \\
  $t_{ij}$ & tempo que qualquer caminhão demora para ir da planta i ao cliente j \\
  $t_{ji}$ & tempo que qualquer caminhão demora para ir do cliente j à planta i \\
  $a_{j}$ & início da janela de entrega do cliente j \\
  $b_{j}$ & fim da janela de entrega do cliente j\\
  $T$ & horizonte de tempo de um dia de trabalho \\
  \hline
  \end{tabular}
  \caption{Lista de Parâmetros}
  \label{tab:parametros}
\end{table}


% \begin{table}[ht!]
%   \centering
%   \begin{tabular}{p{3cm}p{10cm}}
%   \toprule
%   \textbf{Variáveis} & \multicolumn{1}{c}{\textbf{Descrição}} \\
%   \midrule
%   $x_{ijlr}\in \{0,1\}$ & variável binária que, quando ativa, indica que o veículo $k \in K$, em sua l-é\-si\-ma via\-gem $l\in L_k$, sai
%   da planta $i\in I$ para o cliente $j\in J$ \\
%   u$_{ik}$ & variável binária que, quando ativa, indica que o veículo k está atribuído à planta i \\
%   s$_{kl}\geq 0$ & momento de início da l-ésima viagem do veículo k \\
%   \bottomrule
%   \end{tabular}
%   \caption{Lista de Variáveis}
%   \label{tab:variables}
% \end{table}

\begin{table}[ht!]
    \centering
    \begin{tabular}{p{3cm}p{10cm}}
    \toprule
    \textbf{Variáveis} & \multicolumn{1}{c}{\textbf{Descrição}} \\
    \midrule
    $x_{i,j,l,r} \in \{0,1\}$ & Variável binária que assume valor 1 se um caminhão atende a viagem $l$ do cliente $j$ \\
    & a partir da planta $i$, terminando no tempo $r$, e 0 caso contrário. \\
    $y_{i,j,l,r} \in \{0,1\}$ & Variável binária que assume valor 1 se um caminhão retorna da viagem $l$ do cliente $j$ \\
    & para a planta $i$ no tempo $r$, e 0 caso contrário. \\
    $z_i \in \mathbb{Z}_+$ & Número de caminhões que iniciam na planta $i$. \\
    \bottomrule
    \end{tabular}
    \caption{Lista de Variáveis de Decisão}
    \label{tab:variables_decisao}
  \end{table}
  

% \subsection*{Índices e Conjuntos}
% \begin{itemize}
%     \item \( i \in I \) : Usinas (plantas)
%     \item \( j \in J \) : Clientes
%     \item \( l \in L_j \) : Viagens do cliente \( j \)
%     \item \( r \in T \) : Instantes de tempo discretizados
% \end{itemize}

% \subsection*{Variáveis de Decisão}
% \begin{align*}
%     x_{i,j,l,r} & \in \{0,1\}, & \text{1 se um caminhão atende a viagem } l \text{ do cliente } j \\ 
%                 &  & \text{ da planta } i \text{ terminando no tempo } r, \text{ 0 caso contrário} \\
%     y_{i,j,l,r} & \in \{0,1\}, & \text{1 se um caminhão retorna da viagem } l \text{ do cliente } j \\ 
%                 &  & \text{ da planta } i \text{ no tempo } r, \text{ 0 caso contrário} \\
%     z_i & \in \mathbb{Z}_+, & \text{Número de caminhões que iniciam na planta } i
% \end{align*}


\subsection*{Função Objetivo}
\begin{equation}
    \max \sum_{i,j,l,r} x_{i,j,l,r} \cdot d_{j,l}
\end{equation}

\subsection*{Restrições}

\textbf{Cada viagem do cliente é atendida no máximo uma vez:}
\begin{equation}
    \sum_{i,r} x_{i,j,l,r} \leq 1, \quad \forall j \in J, \forall l \in L_j
\end{equation}

\textbf{Restrições de intervalo de tempo entre viagens:}
\begin{equation}
    \sum_{i,r} r x_{i,j,l+1,r} - \sum_{i,r} (r + T_{unload}) x_{i,j,l,r} \geq 0, \quad \forall j \in J, \forall l \in L_j
\end{equation}
\begin{equation}
    \sum_{i,r} r x_{i,j,l+1,r} - \sum_{i,r} (r + T_{unload}) x_{i,j,l,r} \leq 5, \quad \forall j \in J, \forall l \in L_j
\end{equation}

\textbf{A execução das viagens deve ser sequencial:}
\begin{equation}
    \sum_{i,r} x_{i,j,l+1,r} = \sum_{i,r} x_{i,j,l,r}, \quad \forall j \in J, \forall l \in L_j
\end{equation}

\textbf{As viagens devem terminar dentro do horizonte de tempo:}
\begin{equation}
    \sum_{i,r} (r + t_{i,j} + T_{unload} + d_{i}) y_{i,j,l,r} \leq T, \quad \forall j \in J, \forall l \in L_j
\end{equation}

\textbf{Janela de tempo de descarregamento:}
\begin{equation}
    \sum_{i,r} (b_j - r - T_{unload}) x_{i,j,l,r} \geq 0, \quad \forall j \in J, \forall l \in L_j
\end{equation}
\begin{equation}
    \sum_{i,r} (r - a_j) x_{i,j,l,r} \geq 0, \quad \forall j \in J, \forall l \in L_j
\end{equation}

\textbf{Limitação da frota disponível:}
\begin{equation}
  \sum_{i,j,l,r \mid r \geq rr \geq r - t_{i,j}} x_{i,j,l,r} +
  \sum_{i,j,l,r \mid r < rr \leq r + t_{i,j} + T_{unload}} y_{i,j,l,r} 
  \leq K, \quad \forall rr \in T
\end{equation}

\textbf{Viagem de ida e volta devem ser consistentes:}
\begin{equation}
    \sum_{i,r} r x_{i,j,l,r} = \sum_{i,r} r y_{i,j,l,r}, \quad \forall j \in J, \forall l \in L_j
\end{equation}

\textbf{Disponibilidade inicial dos veículos nas plantas:}
\begin{equation}
  \sum_{j,l,r \mid r + t_{i,j} + T_{unload} \leq rr} y_{i,j,l,r} + z_i -
  \sum_{j,l,r \mid r - t_{i,j} < rr} x_{i,j,l,r} \geq 0, \quad \forall i \in I, \forall rr \in T
\end{equation}

\textbf{Limite da frota total:}
\begin{equation}
    \sum_{i} z_i \leq K
\end{equation}
