\chapter{Introdução}
Como um indicador geral do desenvolvimento da sociedade, o Produto Interno Bruto (PIB)  provê informações que estão intimamente relacionadas ao bem-estar dos países \citep{gdpValidity}. A tendência internacional é que o PIB cresça ao longo dos anos \citep{oecd2024realgdp}. Como um dos setores que mais influenciam no PIB, a Indústria de Construção Civil (ICC) assume, em média, 13\% do PIB mundial \citep{mckinsey,fontereserva}.
No mundo, o desempenho da ICC mantém, em geral, um alto nível de correlação com o consumo de cimento \citep{globbulk}. No Brasil, a participação da ICC no PIB nacional é de 3,2\% \citep{SNIC2022}. 

Apesar de imensa e de grande relevância internacional, a indústria cimenteira lida com inúmeros desafios no que tange a velocidade de entrega, o atendimento de grandes demandas, a garantia de qualidade e o desperdício. Acompanhando a tendência de crescimento do PIB internacional, estes desafios também estão propensos a crescer e se tornar cada vez mais relevantes, principalmente quando considerados fatores como o esgotamento dos recursos naturais e a sustentabilidade, dado que a o consumo de cimento está também fortemente vinculado ao alto nível de consumo de água e a alta emissão de CO$_2$ \citep{Watts2019Concrete}.

Minimizar o desperdício é essencial para mitigar os impactos sociais e ambientais advindos do consumo de cimento. O presente projeto se concentra no desenvolvimento de estratégias para aumentar a eficiência na ICC. Especificamente, aborda desafios relacionados com a entrega de concreto, um componente crítico em projetos de construção, descrito por \cite{kinable}. Ao otimizar os processos de entrega de concreto, visamos ter um aumento no nível de serviço, redução da emissão de $CO_2$ e uma racionalização das operações logísticas de uma empresa cimenteira.

O problema de entrega com caminhões betoneira é um componente essencial da ICC e envolve a distribuição de concreto pronto a um conjunto de clientes com demandas específicas. Esse atendimento é realizado por meio de múltiplas plantas produtoras, respeitando restrições de recursos, como a quantidade de veículos disponíveis, a capacidade de produção das plantas e as janelas de horário de recebimento dos clientes. O uso de diversas plantas justifica-se pela natureza perecível do concreto, visando a otimização do tempo de atendimento e a garantia de qualidade do produto entregue.

O presente trabalho utiliza a pesquisa de \cite{cantu} como referência, com o intuito de fazer implementar sua formulação matemática do problema de roteamento periódico de veículos multi-plantas com datas de vencimento e janelas de tempo e sua pesquisa adaptativa aleatória gananciosa reativa. O uso é feito adotando  simplificações do problema, que serão discorridas nos capítulos seguintes. Posteriormente, serão analisados os limites inferiores do modelo matemático através das Relaxações Lagrangiana e Linear e também serão feitas análises comparativas aprimoramentos e novas abordagens para o encontro dos limites superiores através de heurísticas.



Como objetivos específicos, propomos a implementação de um modelo matemático para abordar o problema de entrega de concreto pronto. Espera-se que, em média, esse modelo forneça soluções superiores aos modelos presentes na literatura, com a qualidade dos resultados confirmada com a reprodução da solução das instâncias da literatura. A superioridade do modelo será avaliada em termos de qualidade das soluções e tempo de execução. Dessa forma, a abordagem proposta poderá contribuir para uma solução mais eficaz e de menor custo computacional, mantendo a acurácia necessária para aplicações práticas.

Como objetivos gerais, visamos contribuir para a literatura avançando em sentidos com lacunas a serem preenchidas. Contribuindo para a criação de referências dos resultados da execução dos modelos, considerando que as pesquisas em torno do problema da entrega de concreto pronto é carente de padrões. Além disso, visamos também contribuir para a validação e valorização dos trabalhos bem estruturados e com bons resultados, que possibilitam a replicabilidade e aplicação. 

\textcolor{blue}{ quais a suas contribuições}

% \textcolor{blue}{ colocar comoe o trabalho está organizado.}

O presente trabalho está organizada em \textcolor{red}{sete} capítulos. A Introdução apresenta o problema, os objetivos e a estrutura do trabalho. Em Revisão de Literatura, exploramos estudos e metodologias relacionadas, situando o trabalho no contexto acadêmico. O capítulo de Notações, Definição e Formulação formaliza o problema matematicamente e introduz as notações utilizadas. No capítulo de Instâncias, descrevemos os dados e justificamos a escolha dos cenários para validação. Em Metodologias de Resolução, detalhamos a Relaxação Lagrangiana e a heurística Multi-Start, abordagens empregadas para otimizar a solução. No capítulo de Resultados, apresentamos e analisamos os resultados obtidos com os métodos propostos, e, por fim, em Conclusão, sintetizamos as contribuições e sugerimos direções para estudos futuros.

